%%%%%%%%%%%%%%%%%%%%%%%%%%%%%%%%%%%%%%%%%%%%%%%%%%%%%%%%%%%%%%%%%%%%%%%%%%%%%%%
% Cap�tulo 2: Fundamentos Te�ricos 
%%%%%%%%%%%%%%%%%%%%%%%%%%%%%%%%%%%%%%%%%%%%%%%%%%%%%%%%%%%%%%%%%%%%%%%%%%%%%%%

%++++++++++++++++++++++++++++++++++++++++++++++++++++++++++++++++++++++++++++++

%En este cap�tulo se han de presentar los antecedentes te�ricos y pr�cticos que
%apoyan el tema objeto de la investigaci�n.

%++++++++++++++++++++++++++++++++++++++++++++++++++++++++++++++++++++++++++++++

\section{Primer apartado del segundo cap�tulo}
\label{2:sec:1}
  La regla del trapecio es un m�todo de integraci�n num�rica, es decir,
  un m�todo para calcular aproximadamente el valor de la integral definida 
  La regla se basa en aproximar el valor de la integral de f(x) por el 
  de la funci�n lineal que pasa a trav�s de los puntos(a,f(a)) y (b,f(b)).
  La integral de �sta es igual al �rea deltrapecio bajo la gr�fica de la
  funci�n lineal. Se sigue que: 

  (FORMULA)

  Y el error es:

   (FORMULA)

  Siendo (Etsilon) un n�mero entre a y b.

\section{Segundo apartado del segundo cap�tulo}
\label{2:sec:2}

 Regla del trapecio compuesta
%\begin{center}
%\includegraphics[width=0.20\textwidth]{images/Regla-Trap-compuesta}\\[0.25cm]
%\end{center}

  La regla del trapecio compuesta o regla de los trapecios es una forma de
  aproximar una integral definida utilizando n trapecios. En la formulaci�n
  de este m�todo se supone que f es continua y positiva en el intervalo [a,b].
  De tal modo la integral definida representa el �rea de la regi�n delimitada
  por la gr�fica de f y el eje x, desde x=a hasta x=b.
  Primero se divide el intervalo [a,b] en n subintervalos, cada uno de ancho
  (Formula)
  Despu�s de realizar todo el poceso matem�tico se llega a la siguiente f�rmula:
  (Formula)
  Donde h= (b-a)/n y n es el numero de divisiones.
  La expresion anterior se puede escribir de la siguiente manera:
  (Formula)



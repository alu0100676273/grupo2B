%%%%%%%%%%%%%%%%%%%%%%%%%%%%%%%%%%%%%%%%%%%%%%%%%%%%%%%%%%%%%%%%%%%%%%%%%%%%%
% Cap�tulo 1: Motivaci�n y Objetivos 
%%%%%%%%%%%%%%%%%%%%%%%%%%%%%%%%%%%%%%%%%%%%%%%%%%%%%%%%%%%%%%%%%%%%%%%%%%%%%%%

%Los objetivos le dan al lector las razones por las que se realiz� el
%proyecto o trabajo de investigaci�n.

%---------------------------------------------------------------------------------
\section{Secci\'on Uno}
\label{1:sec:1}
\parindent=0.5cm
\raggedright
El objetivo de este trabajo es realizar una integral mediante la Regla del Trapecio. La Regla
del Trapecio resulta de mucha utilidad dado que es una forma muy sencilla de saber el valor de
la integral definida.
\begin{figure}[!th]
\begin{center}
\includegraphics[width=0.25\textwidth]{images/Regla-Trapecio.eps}
\end{center}
\end{figure}
  
%---------------------------------------------------------------------------------
\section{Secci\'on Dos}
\label{1:sec:2}
\parindent=0.5cm
\raggedright
Como funci�n vamos a utilizar:
  \[
  f(x)=\frac{1}{sqrt(2\pi)} \quad\text{e}^{\frac{-x^2}{2}}
  \]
en el itervalo [-1,1]
\begin{itemize}
  \item Item 1
  \item Item 2
  \item Item 3
\end{itemize}


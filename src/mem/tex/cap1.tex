%%%%%%%%%%%%%%%%%%%%%%%%%%%%%%%%%%%%%%%%%%%%%%%%%%%%%%%%%%%%%%%%%%%%%%%%%%%%%
% Capítulo 1: Motivación y Objetivos 
%%%%%%%%%%%%%%%%%%%%%%%%%%%%%%%%%%%%%%%%%%%%%%%%%%%%%%%%%%%%%%%%%%%%%%%%%%%%%%%

%Los objetivos le dan al lector las razones por las que se realizó el
%proyecto o trabajo de investigación.

%---------------------------------------------------------------------------------
\section{Motivaciones y objetivos}
\label{1:sec:1}

El objetivo principal de este trabajo es realizar una integral mediante la Regla del Trapecio y ver
su veracidad. La Regla del Trapecio resulta de mucha utilidad dado que es una forma muy sencilla de
aproximar el valor de la integral definida entre dos puntos a y b. Geométricamente, es equivalente a
aproximar el área del  trapezoide bajo la recta que conecta f(a) y f(b).

%---------------------------------------------------------------------------------
\section{Uso de una funci\'on de estudio}
\label{1:sec:2}

Para la realización de este trabajo emplearemos la función:
\[
f(x)=\frac{1}{\sqrt(2\pi)} \quad\text{e}^{\frac{-x^2}{2}}
\]
comprendida en el itervalo [-1,1]

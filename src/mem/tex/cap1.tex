%%%%%%%%%%%%%%%%%%%%%%%%%%%%%%%%%%%%%%%%%%%%%%%%%%%%%%%%%%%%%%%%%%%%%%%%%%%%%
% Cap�tulo 1: Motivaci�n y Objetivos 
%%%%%%%%%%%%%%%%%%%%%%%%%%%%%%%%%%%%%%%%%%%%%%%%%%%%%%%%%%%%%%%%%%%%%%%%%%%%%%%

%Los objetivos le dan al lector las razones por las que se realiz� el
%proyecto o trabajo de investigaci�n.

%---------------------------------------------------------------------------------
\section{Primera Secci\'on}
\label{1:sec:1}
\parindent=0.5cm
\raggedright
El objetivo de este trabajo es realizar una integral mediante la Regla del Trapecio. La Regla
del Trapecio resulta de mucha utilidad dado que es una forma muy sencilla de aproximar el valor de
la integral definida entre dos puntos a y b.Geom�tricamente, es equivalente a aproximar el �rea del 
trapezoide bajo la l�nea recta que conecta f(a) y f(b).
\begin{center}
\includegraphics[width=0.25\textwidth]{images/Regla-Trapecio}
\end{center}

%---------------------------------------------------------------------------------
\section{Segunda Secci\'on}
\label{1:sec:2}
\parindent=0.5cm
\raggedright
Para la realizaci�n de este trabajo emplearemos la funci\'on:
\[
  f(x)=\frac{1}{\sqrt(2\pi)} \quad\text{e}^{\frac{-x^2}{2}}
\]
comprendida en el itervalo [-1,1]

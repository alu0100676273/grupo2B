%%%%%%%%%%%%%%%%%%%%%%%%%%%%%%%%%%%%%%%%%%%%%%%%%%%%%%%%%%%%%%%%%%%%%%%%%%%%%
% Capítulo 4: Conclusiones y Trabajos Futuros 
%%%%%%%%%%%%%%%%%%%%%%%%%%%%%%%%%%%%%%%%%%%%%%%%%%%%%%%%%%%%%%%%%%%%%%%%%%%%%%%
\parindent=1cm
\raggedright
Podemos sacar muchas conclusiones de la aplicación de la integración por trapecio a esta función en particular.
Primeramente, cabe destacar que esta función es par, y por ello se simplifica mucho la operatoria a la hora de llevar a cabo todos los métodos.
En cuanmto a los análisis de los experimentos, nos dan información necesaria para sacar varias conclusiones. Como se esperaba, la integración por trapecio simple
es bastante más ineficiente que la integración por trapecio compuesta, ya que con sólo tomar cuatro divisiones del intervalo, ya se acerca bastante a la integral
definida .Por tanto, si escogemos un número de divisiones algo elevado, nuestra medida del área que se abarca va a ser aproximadamente acertada. 
Sin embargo, la integración por trapecio simple es un método que posee un error demasiado grande, que impide que nos podamos fiar de esta regla.
 No obstante, disponemos de una función que nos devuelve un error posible, y hemos comprobado que se cumple, en los experimentos llevados a cabo.
También la función del error es par, con lo que no era necesario tomar valores entre -1 y 1, sino que bastaba con tomarlos entre 0 y 1. Asimismo, igualando la
derivada de dicha función cero, pudimos obtener un error máximo, que es muy útil ya que el método del trapecio simple no es muy exacto. Con ello se consigue una cota 
superior e inferior de la integral definida, por lo que finalmente se le encuentra una gran utilidad a esta regla; debemos recordar que la regla del
trapecio compuesta no tiene un valor que nos pueda definir el error producido.

%%%%%%%%%%%%%%%%%%%%%%%%%%%%%%%%%%%%%%%%%%%%%%%%%%%%%%%%%%%%%%%%%%%%%%%%%%%%%%%
% Capítulo 2: Fundamentos Teóricos 
%%%%%%%%%%%%%%%%%%%%%%%%%%%%%%%%%%%%%%%%%%%%%%%%%%%%%%%%%%%%%%%%%%%%%%%%%%%%%%%

%++++++++++++++++++++++++++++++++++++++++++++++++++++++++++++++++++++++++++++++

%En este capítulo se han de presentar los antecedentes teóricos y prácticos que
%apoyan el tema objeto de la investigación.

%++++++++++++++++++++++++++++++++++++++++++++++++++++++++++++++++++++++++++++++
\section{Regla del Trapecio}
\label{2:sec:1}
\parindent=0.5cm
\raggedright
Este método integración numérica calcula aproximadamente el valor de la integral definida. 
Se aproxima el valor de la integral de f(x) por el de la función lineal que pasa 
a través de los puntos(a,f(a)) y (b,f(b)). La integral de esta es igual al área del 
trapecio bajo la gráfica de la función lineal. Se sigue que: 
\[
\int_{a}^{b} f(x)dx \approx\left(b-a\right)\frac{f(a)+f(b)}{2} 
\]
Una estimación para el error de truncamiento local de una sola aplicación de la regla trapezoidal es:
\[
-\frac{\left(b-a\right)^3}{12}  \displaystyle f^{(2)}(\epsilon)
\]
Perteneciendo $\epsilon$ al intervalo entre a y b.


\parindent=0.5cm
\raggedright
Una manera de mejorar la exactitud es utilizar la regla del trapecio compuesta, es decir, dividir
el intervalo de integración entre a y b en n subintervalos, cada uno de ancho
\[
\bigtriangleup{x}=\frac{b-a}{n}
\]
y aplicar el método a cada uno de ellos. Las ecuaciones resultantes son llamadas fórmulas
de integración de múltiple aplicación o compuestas.
\[
\int_{a}^{b} f(x)dx \sim\frac{h}{2}\left[f(a) + 2f(a+h) + 2f(a+2h) + ... + f(b)\right]
\]
Donde 
\[
h=\frac{b-a}{n} 
\]
La expresión anterior se puede escribir de la siguiente manera:
\[
\int_{a}^{b} f(x)dx \sim\frac{b-a}{n}\left(\frac{f(a)+f(b)}{2} + \sum_{k=1}^{n-1} a+k\frac{b-a}{n} \right)
\]

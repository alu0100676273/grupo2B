%%%%%%%%%%%%%%%%%%%%%%%%%%%%%%%%%%%%%%%%%%%%%%%%%%%%%%%%%%%%%%%%%%%%%%%%%%%%%%%
% Cap�tulo 2: Fundamentos Te�ricos 
%%%%%%%%%%%%%%%%%%%%%%%%%%%%%%%%%%%%%%%%%%%%%%%%%%%%%%%%%%%%%%%%%%%%%%%%%%%%%%%

%++++++++++++++++++++++++++++++++++++++++++++++++++++++++++++++++++++++++++++++

%En este cap�tulo se han de presentar los antecedentes te�ricos y pr�cticos que
%apoyan el tema objeto de la investigaci�n.

%++++++++++++++++++++++++++++++++++++++++++++++++++++++++++++++++++++++++++++++

\label{2:sec:1}
\parindent=0.5cm
\raggedright
  La regla del trapecio es un m�todo de integraci�n num�rica, es decir,
  un m�todo para calcular aproximadamente el valor de la integral definida 
  La regla se basa en aproximar el valor de la integral de f(x) por el 
  de la funci�n lineal que pasa a trav�s de los puntos(a,f(a)) y (b,f(b)).
  La integral de �sta es igual al �rea deltrapecio bajo la gr�fica de la
  funci�n lineal. Se sigue que: 
  \[
  \int_{a}^{b} f(x)dx \approx\left(b-a\right)\frac{f(a)+f(b)}{2} 
  \]
  Y el error es: 
   \[
   -\frac{\left(b-a\right)^3}{12}  \displaystyle f^{(2)}(\epsilon)
   \]
  Siendo $\epsilon$ un nmero entre a y b.

\section{Regla del Trapecio compuesta}
\label{2:sec:2}
\begin{figure}[!th]
\begin{center}
\includegraphics[width=0.25\textwidth]{images/Regla-Trap-compuesta}
\end{center}
\end{figure}
\parindent=0.5cm
\raggedright
  La regla del trapecio compuesta o regla de los trapecios es una forma de
  aproximar una integral definida utilizando n trapecios. En la formulaci�n
  de este m�todo se supone que f es continua y positiva en el intervalo [a,b].
  De tal modo la integral definida representa el �rea de la regi�n delimitada
  por la gr�fica de f y el eje x, desde x=a hasta x=b.
  Primero se divide el intervalo [a,b] en n subintervalos, cada uno de ancho
  \[
   \bigtriangleup{x}=\frac{b-a}{n}
  \]
  Despu�s de realizar todo el poceso matem�tico se llega a la siguiente f\'ormula:
  \[
  \int_{a}^{b} f(x)dx \sim\frac{h}{2}\left[f(a) + f(a+h) + f(a+2h) + ... + f(b)\right]
  \]
  
  Donde 
  \[
    h=\frac{b-a}{n} 
  \]
  y n es el numero de divisiones.
  La expresion anterior se puede escribir de la siguiente manera:
  \[
  \int_{a}^{b} f(x)dx \sim\frac{b-a}{n}\left(\frac{f(a)+f(b)}{2} + \sum_{k=1}^{n-1} a+k\frac{b-a}{n} \right)
  \]



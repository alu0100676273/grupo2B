%%%%%%%%%%%%%%%%%%%%%%%%%%%%%%%%%%%%%%%%%%%%%%%%%%%%%%%%%%%%%%%%%%%%%%%%%%%%%%%
% Cap�tulo 2: Fundamentos Te�ricos 
%%%%%%%%%%%%%%%%%%%%%%%%%%%%%%%%%%%%%%%%%%%%%%%%%%%%%%%%%%%%%%%%%%%%%%%%%%%%%%%

%++++++++++++++++++++++++++++++++++++++++++++++++++++++++++++++++++++++++++++++

%En este cap�tulo se han de presentar los antecedentes te�ricos y pr�cticos que
%apoyan el tema objeto de la investigaci�n.

%++++++++++++++++++++++++++++++++++++++++++++++++++++++++++++++++++++++++++++++
\section{Regla del Trapecio}
\label{2:sec:1}
\parindent=0.5cm
\raggedright
Este m�todo integraci�n num�rica calcula aproximadamente el valor de la integral definida. 
Se aproxima el valor de la integral de f(x) por el de la funci�n lineal que pasa 
a trav�s de los puntos(a,f(a)) y (b,f(b)). La integral de �sta es igual al �rea del 
trapecio bajo la gr�fica de la funci�n lineal. Se sigue que: 
\[
\int_{a}^{b} f(x)dx \approx\left(b-a\right)\frac{f(a)+f(b)}{2} 
\]
Una estimaci�n para el error de truncamiento local de una sola aplicaci�n de la regla trapezoidal es:
\[
-\frac{\left(b-a\right)^3}{12}  \displaystyle f^{(2)}(\epsilon)
\]
Perteneciendo $\epsilon$ al intervalo entre a y b.

\section{Regla del Trapecio compuesta}
\label{2:sec:2}
\begin{figure}[!th]
\begin{center}
\includegraphics[width=0.25\textwidth]{images/Regla-Trap-compuesta}
\end{center}
\end{figure}
\parindent=0.5cm
\raggedright
Una manera de mejorar la exactitud es utilizar la regla del trapecio compuesta, es decir, dividir
el intervalo de integraci�n entre a y b en n subintervalos, cada uno de ancho
\[
\bigtriangleup{x}=\frac{b-a}{n}
\]
y aplicar el m�todo a cada uno de ellos. Las ecuaciones resultantes son llamadas f�rmulas
de integraci�n de m�ltiple aplicaci�n o compuestas.
\[
\int_{a}^{b} f(x)dx \sim\frac{h}{2}\left[f(a) + f(a+h) + f(a+2h) + ... + f(b)\right]
\]
Donde 
\[
h=\frac{b-a}{n} 
\]
La expresion anterior se puede escribir de la siguiente manera:
\[
\int_{a}^{b} f(x)dx \sim\frac{b-a}{n}\left(\frac{f(a)+f(b)}{2} + \sum_{k=1}^{n-1} a+k\frac{b-a}{n} \right)
\]



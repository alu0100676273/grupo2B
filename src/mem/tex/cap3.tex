%%%%%%%%%%%%%%%%%%%%%%%%%%%%%%%%%%%%%%%%%%%%%%%%%%%%%%%%%%%%%%%%%%%%%%%%%%%%%%%
% Cap�tulo 3: Procedimiento experimental 
%%%%%%%%%%%%%%%%%%%%%%%%%%%%%%%%%%%%%%%%%%%%%%%%%%%%%%%%%%%%%%%%%%%%%%%%%%%%%%%

%Este cap�tulo ha de contar con seccciones para la descripci�n de los experimentos 
%y del material.

%Tambi�n debe haber una secci�n para los resultados obtenidos y una �ltima de 
%an�lisis de los resultados.

%++++++++++++++++++++++++++++++++++++++++++++++++++++++++++++++++++++++++++++++
\section{Descripci�n de los experimentos}
\label{3:sec:1}
\parindent=1cm
\raggedright
El experimento llevado a cabo es la realizaci�n de la integral definida y la aproximaci�n
mediante la regla del trapecio y la regla del trapecio compuesta. Adem�s de conocer el error
de truncamiento que existe. Veamos\\
Integral definida 
\[
\int_{-1}^{1} \frac{1}{sqrt(2\pi)} \quad\text{e}^{\frac{-x^2}{2}}dx\approx0.682689 
\]
Regla del Trapecio
\[
\int_{-1}^{} \frac{1}{sqrt(2\pi)} \quad\text{e}^{\frac{-x^2}{2}}dx\approx\left(1-(-1)a\right)\frac{f(-1)+f(1)}{2}
\]
Regla del Trapecio compuesta con n=4
\[
h=\frac{1-(-1)}{4} =\frac{1}{2} 
\]
\[
\int_{-1}^{1} \frac{1}{sqrt(2\pi)} \quad\text{e}^{\frac{-x^2}{2}}dx\approx\left[f(-1) + f(\frac-{1}{2}) + f(0) + f(\frac{1}{2} + f(1)\right]
\]

Tomando $\Etsilon$ con un valor de   , el error de truncamiento es:


%++++++++++++++++++++++++++++++++++++++++++++++++++++++++++++++++++++++++++++++
\section{Descripci�n del material}
\label{3:sec:2}
\parindent=1cm
\raggedright
El material que hemos empleado para la realizaci�n de este proyecto es el lenguaje de
programaci�n Python, para la creaci�n de los programas que verifican este m�todo de integraci�n, 
y el sistema de composici�n de textos \LaTeX para el dise�o del proyecto.


%++++++++++++++++++++++++++++++++++++++++++++++++++++++++++++++++++++++++++++++
\section{Resultados obtenidos}
\label{3:sec:3}
\parindent=1cm
\raggedright
Los resultados obtenidos son:
Integral definida 
\[
\int_{-1}^{1} \frac{1}{\sqrt(2\pi)} \quad\text{e}^{\frac{-x^2}{2}}dx\approx0.682689 
\]
Regla del Trapecio
\[
\int_{-1}^{1} \frac{1}{\sqrt(2\pi)} \quad\text{e}^{\frac{-x^2}{2}}dx\approx\left(1-(-1)\right)\frac{f(-1)+f(1)}{2}=0.483941
\]
Regla del Trapecio compuesta con n=4
\[
\int_{-1}^{1} \frac{1}{\sqrt(2\pi)} \quad\text{e}^{\frac{-x^2}{2}}dx\approx\left[f(-1) + f(\frac-{1}{2}) + f(0) + f(\frac{1}{2} + f(1)\right]\approx=0.882385
\]
%------------------------------------------------------------------------------
%\begin{figure}[!th]
%\begin{center}
%\includegraphics[width=0.75\textwidth]{images/figura1.eps}
%\caption{Ejemplo de figura}
%\label{fig:1}
%\end{center}
%\end{figure}
%------------------------------------------------------------------------------
Resultados obtenidos en tiempo y velocidad para la regla del trapecio simple y la regla del trapecio compuesta
%--------------------------------------------------------------------------
\begin{table}[!ht]
\begin{center}
\begin{tabular}{|c|c|} \hline 
\textbf{Tiempo por Trapecio simple} & \textbf{Tiempo por trapecio compuesto} \\ 
\textbf{(en s)} & \textbf{(en s)} \\ \hline \hline
0.1412169 &
0.1414737
\\
\hline

0.1414649 &
0.1415798
\\
\hline

0.1412210 &
0.1413900
\\
\hline


\end{tabular}
\end{center}
\caption{Resultados experimentales de tiempo (s) y velocidad (m/s)}
\label{tab:1}
\end{table}


%------------------------------------------------------------------------------

%++++++++++++++++++++++++++++++++++++++++++++++++++++++++++++++++++++++++++++++
\section{An�lisis de los resultados}
\label{3:sec:4}
\parindent=1cm
\raggedright
 

